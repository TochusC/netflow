\section{程序流程}

NetFlow 的运行流程涉及 Electron 主进程、渲染进程以及外部 Python 嗅探进程之间的紧密协作。

\begin{enumerate}
  \item \textbf{系统初始化}:应用启动时,\texttt{main.js} 初始化 Electron 主进程,创建主窗口,并实例化 \texttt{PluginManager} 和 \texttt{DockingManager}。
  \item \textbf{插件加载}:\texttt{PluginManager} 扫描 \texttt{plugins/} 目录,读取 \texttt{plugin.json} 配置,根据插件类型(Sniffer, Visualizer, Reporter)将其注册到系统中。
  \item \textbf{界面渲染}:渲染进程加载 \texttt{index.html},初始化 Docking 布局,并向主进程请求已加载的插件 UI 资源。
  \item \textbf{嗅探器启动}:主进程通过 \texttt{child\_process} 模块生成 Python 子进程 (\texttt{sniffer.py}),用于底层数据包捕获。
  \item \textbf{数据捕获与传输}:
    \begin{itemize}
        \item Python 进程使用原始套接字 (Raw Socket) 捕获网络数据包。
        \item 解析数据包头部信息,并通过标准输出 (Stdout) 发送。
        \item 主进程监听 Stdout,接收数据并通过 IPC 通道发送至嗅探器插件。
        \item 嗅探器插件接收数据,进行必要的预处理后,将数据添加至DOM全局变量,并广播给所有注册的可视化插件。
    \end{itemize}
  \item \textbf{可视化与交互}:各可视化插件接收数据流,更新内存中的数据结构,并驱动 Canvas 或 DOM 进行实时渲染。
\end{enumerate}

\vspace{1cm}

\begin{figure}[ht]
\centering
\begin{tikzpicture}[node distance=1.6cm]

% Nodes
\node (init) [startstop] {系统初始化 (Main)};
\node (load) [process, below of=init] {加载插件 (PluginManager)};
\node (render) [process, below of=load] {渲染主界面 (Renderer)};
\node (start) [io, below of=render] {启动嗅探 (User)};
\node (python) [process, below of=start] {启动 Python 进程};
\node (capture) [process, below of=python] {Raw Socket 捕获 \& 解析};
\node (main_relay) [process, below of=capture] {Stdout 接收 \& IPC 转发 (Main)};
\node (sniffer_logic) [process, below of=main_relay] {嗅探插件处理 (Renderer)};
\node (visual_update) [process, below of=sniffer_logic] {可视化插件渲染};

% Arrows
\draw [arrow] (init) -- (load);
\draw [arrow] (load) -- (render);
\draw [arrow] (render) -- (start);
\draw [arrow] (start) -- (python);
\draw [arrow] (python) -- (capture);
\draw [arrow] (capture) -- node[anchor=west] {Stdout} (main_relay);
\draw [arrow] (main_relay) -- node[anchor=west] {IPC} (sniffer_logic);
\draw [arrow] (sniffer_logic) -- node[anchor=west] {DOM/Event} (visual_update);
\draw [arrow] (visual_update) -- ++(4.0,0) |- node[anchor=south] {实时刷新} (render);

\end{tikzpicture}
\caption{NetFlow 系统运行流程图}
\label{fig:flowchart}
\end{figure}
