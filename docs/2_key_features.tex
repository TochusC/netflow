\section{关键功能实现(抓包与插件生态)}
本节详细说明抓包模块、插件生态策略、数据管线与 Docking 系统的实现细节,便于教师评估设计的技术深度。

\subsection{抓包实现(Python Sniffer)}
抓包模块的核心是一个独立运行的 Python 脚本(\texttt{sniffer.py}),它通过原始套接字(Raw Socket)直接与网络接口交互。

\subsubsection{底层捕获机制}
使用了 Python 的 \texttt{socket} 库,配置为 \texttt{AF\_INET} 和 \texttt{SOCK\_RAW},并设置 \texttt{IP\_HDRINCL} 选项以包含 IP 头。在 Windows 平台上,还需要调用 \texttt{ioctl(SIO\_RCVALL)} 以启用混杂模式,捕获流经网卡的所有数据包。

\begin{lstlisting}[language=Python, caption=Python Raw Socket 初始化]
self.socket = socket.socket(socket.AF_INET, socket.SOCK_RAW, socket.IPPROTO_IP)
self.socket.bind((self.local_ip, 0))
self.socket.setsockopt(socket.IPPROTO_IP, socket.IP_HDRINCL, 1)
self.socket.ioctl(socket.SIO_RCVALL, socket.RCVALL_ON)
\end{lstlisting}

\subsubsection{协议解析与过滤}
捕获到的二进制数据包首先被解析 IP 头(前20字节),提取源/目的 IP 和协议类型。如果是 TCP/UDP,进一步解析端口号以推断应用层协议(如 HTTP, DNS)。
过滤器实现了一个基于 \texttt{eval} 的动态求值引擎。用户输入的过滤表达式(如 \texttt{src 192.168.1.1 and proto TCP})被转换为 Python 表达式,并在包含 \texttt{src\_ip}, \texttt{dst\_ip} 等变量的上下文中执行。这种方式提供了极高的灵活性,支持复杂的逻辑组合。

\subsubsection{进程间通信 (IPC)}
Python 进程将解析后的数据包以结构化的文本格式输出到标准输出(stdout)。Electron 主进程通过 Node.js 的 \texttt{child\_process.spawn} 启动该脚本,并监听其 stdout 数据流,按行缓冲并解析为 JSON 对象,最后通过 IPC 广播给发送回渲染进程的嗅探器插件,实现数据的实时传输。
嗅探器插件接收数据,进行必要的预处理后,将数据添加至DOM全局变量,并广播给所有注册的可视化插件。

\subsection{数据流与广播(Data Pipeline)}
数据从捕获到展示经历了以下阶段:
\begin{enumerate}
    \item \textbf{采集层}:Python 进程捕获原始字节,解析出关键字段(五元组、长度、Payload),并进行初步过滤。
    \item \textbf{传输层}:主进程通过 Stdout 接收文本流,反序列化为 JavaScript 对象,并附加时间戳。
    \item \textbf{接收层}:主进程通过 \texttt{packet-data-batch} 频道将数据包数组广播发送到所有渲染器中的嗅探器插件。
    \item \textbf{处理层}:嗅探器插件接收数据,进行预处理并存储于全局DOM变量,广播给所有注册的可视化插件。
    \item \textbf{展示层}:各插件根据自身需求处理数据。例如,Visualizer 将数据存入内存数组并更新 UI,Grapher 将数据聚合到时间桶中。
\end{enumerate}

下面的流程图展示了 Netflow 的多层架构与插件环境:

\begin{figure}[ht]
\centering
\resizebox{\textwidth}{!}{%
\begin{tikzpicture}[node distance=15mm]
  % Nodes
  \node (main) [startstop, align=center] {Electron 主进程 (Main)\\ \footnotesize (插件加载 \& IPC 枢纽)};
  \node (sniffer) [process, below=of main, align=center] {Python 抓包插件\\ \footnotesize (外部子进程 / Sniffer)};
  
  \node (renderer) [startstop, right=of main, xshift=4cm, align=center] {Electron 渲染进程 (Renderer)\\ \footnotesize (UI \& 展示层)};
  \node (docking) [process, below=of renderer, align=center] {Docking 停靠系统\\ \footnotesize (插件容器)};
  
  % Visualizer Plugins
  \node (grapher) [process, below=of docking, align=center, minimum width=2.5cm] {流量图表\\ \footnotesize (Traffic Grapher)};
  \node (packet) [process, left=of grapher, align=center, minimum width=2.5cm] {数据包可视化\\ \footnotesize (Packet Visualizer)};
  \node (reporter) [process, right=of grapher, align=center, minimum width=2.5cm] {网络报告\\ \footnotesize (Network Reporter)};

  % Edges
  \draw [arrow] (main) -- node[left, font=\footnotesize] {启动 (Spawn)} (sniffer);
  \draw [arrow] (sniffer) -- node[right, font=\footnotesize] {标准输出 (Stdout)} (main);
  \draw [arrow] (main) -- node[above, font=\footnotesize] {IPC (数据转发)} (renderer);
  \draw [arrow] (renderer) -- (docking);
  
  % Edges to plugins
  \draw [arrow] (docking) -- (packet);
  \draw [arrow] (docking) -- (grapher);
  \draw [arrow] (docking) -- (reporter);
  
  % Control flow
  \draw [arrow, dashed] (docking) to[bend left=20] node[below, font=\footnotesize] {IPC (控制/设置)} (main);

\end{tikzpicture}
}
\caption{Netflow 架构图(进程分离与数据流向)}
\end{figure}

\subsection{插件生态与安全边界}
为保证稳定与可测性,插件运行在严格的Electron沙箱环境中。每个插件通过其 \texttt{plugin.json} 声明所需要加入的脚本代码,包括主进程入口、渲染进程 Preload 脚本和 UI 资源。Preload 脚本负责在渲染器中暴露受限的 API,允许插件与主进程通过 IPC 通信,但禁止直接访问 Node.js 或 DOM,防止恶意代码执行。

\subsection{Docking 与 Popout 系统}
项目实现了可停靠(docking)与弹出窗口(popout)能力,核心实现点:

\begin{itemize}
  \item Docking 容器:渲染层提供一个 Docking 管理组件(\texttt{renderer/docking.js} + \texttt{docking.css}),支持面板的拖拽、停靠与拆分。
  \item Popout 窗口:插件可声明 \texttt{ui.popout = true},由宿主通过 \texttt{BrowserWindow} 新建窗口加载该插件的 \texttt{ui.html},并保留与主进程的双向 IPC。
\end{itemize}
